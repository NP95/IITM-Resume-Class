\documentclass{mycv}
\author{Ashutosh Kumar Jha}
\RollNumber{ME14B148}
\Address{3012, Alakananda Hostel, IIT Madras}
\PhoneNumber{+919790464394}
\Email{ashutosh1296@gmail.com}

\begin{document}
\maketitle
\section{Education}
\begin{EducationTable}
  \EduDetails{B.Tech (Hons.) in Mechanical Engineering}{Indian Institute of Technology Madras, Chennai}{9.19}{2018(expected)}
  \EduDetails{XII}{Mahathi Junior College, Visakhapatnam}{96.7 \%}{2014}
  \EduDetails{X}{Delhi Public School, Visakhapatnam}{10.0}{2012}
\end{EducationTable}
\section{Academic Achievements}

\Point{Was among the \textbf{28 students out of the 840} students at IITM who were awarded a Branch Change at the end of the first semester based on their GPA's. Had the \textbf{highest} GPA among the people who changed to Mechanical Engineering.}

\Point{Among the top 0.1\% in IIT­-JEE(Advanced) and IIT-JEE(Mains),2014.}

\Point{Topped the West Bengal Joint Entrance Examination (WBJEE 2014) with a rank of \textbf{9} out of 80,000 students.}

\Point{Was among the top \textbf{300} students in the country who were awarded the Kishore Vaigyanik Protsahan Yojana (KVPY) fellowship by MHRD in 2012.}

\Point{Ranked \textbf{94} All-India (Zonal Rank - \textbf{2}) out of 50000+ students in the FIITJEE Talent Reward Examination (FTRE) 2012; awarded a cash prize of INR 80000.}

\Point{Was among the \textbf{top 244} students in the country who qualified for the Indian National Olympiad in Informatics (INOI) for selection of the Indian Team for International Informatics Olympiad (IOI) in 2011.}

\section{Publications}
\subsection{A quick reduction approach to extracting regions from Spatio-Temporal Graphs}
\Point{Acknowledged in the paper for creating the graph search algorithms}
\Point{Submitted to Journal of Discrete Algorithms (pending acceptance)}
\section{Internships and Projects}
\datedsubsection{Internship at TITAN Co.}{May - Jul 2016}
\Point{Ongoing}

\datedsubsection{Swarm Robotics with Quadrupeds}{May - Jul 2016}
\Point{Ongoing}
\Point{Presently working on a project with Prof. P.V.Manivannan}

\datedsubsection{Characteristics of an Inverted Pendulum}{Feb - Apr 2016}
\Point{Worked with Prof. S.Sathyan during the semester to design an Inverted Pendulum Bot using Arduino as a part of the course on Instrumentation and Control.}
\Point{The characteristics of the bot were studied and modelled in Simulink. Made use of PID control as well as State-Space control to model the behaviour of the bot under disturbances.}
\Point{Was among the \textbf{best 3} bots out of 40 others which balanced in presence of small disturbances. Was among the only \textbf{5} working bots}
\Point{The bot was kept by the professor as a model bot for the upcoming batches.}

\datedsubsection{Design of Mechanisms using MATLAB}{Jan - Apr 2016}
\Point{Worked with Prof. P.Chandramouli during the semester to model various mechanisms discussed during the classes of Kinematics and Dynamics of Machinery as a project independant from the course.}
\Point{The implementation of Kinematics was done completely in MATLAB. The stress analysis was done using Abaqus.}
\Point{Also performed a case study kinematic analysis of a JCB Backhoe Mechanism. Proposed ideas on actuation of the mechanism which would provide a smoother response of the system}

\datedsubsection{Design of Torsion resistant buildings}{Jul 2015 - Nov 2015}
\Point{Worked with Prof. A.Arockiarajan during the semester to design a method to reduce stresses on roof-tops of buildings in cyclonic areas.}
\Point{Used Abaqus to model dome ­shaped buildings and performed Finite Element Analysis to calculate the stresses developed due to torsion provided by 
the winds.}
\Point{Proposed a method to reduce the stresses by introducing projections of parabolic shapes on the surface 
of the dome. A reduction in stresses by 30\% was observed on implementing the idea.}

\datedsubsection{Designing a Pen Gun Assembly using CREO}{Oct 2015}
\Point{Used CREO Parametric to design various parts of a military pen­-gun tranquilizer as a part of the Machine Drawing course. Completed the assembly of the same.}
\Point{Also performed a stress analysis of the top cap of the pen gun using Abaqus to make sure it doesn't break while shooting.}

\datedsubsection{A touchless tracking interface}{Jan - May 2015}
\Point{A project done under the Electronics Club in the Centre For Innovation, IIT Madras.}
\Point{Developed a touch­less tracking interface using 3 aluminium wrapped cardboard plates and an Arduino UNO to track the position of a human hand \textbf{without using any form of Image Processing}​. Made use of capacitive positioning. Also developed a 3D Tic-Tac-Toe game using the Processing package and used the touch­less interface for input to play the game.}

\section{Courses Completed}
\subsection{\textbf{Mechanical Engineering Courses}}
\begin{Course}
  \Point{Engineering Drawing}
  \Point{Concepts in Engineering Design}
  \Point{Strength of Materials}
  \Point{Basic Electrical Engineering}
  \Point{Kinematics and Dynamics of Machinery}
  \Point{Materials and Design​}
  \Point{Machine Drawing}
  \Point{Engineering Mechanics}
  \Point{Thermodynamics}
  \Point{Foundations of Fluid Mechanics}
  \Point{Instrumentation and Control}
  \Point{Manufacturing Technology}
\end{Course}

\subsection{\textbf{Mathematics and Computer Science Courses}}
\begin{Course}
  \Point{Computational Engineering}
  \Point{Graph Theory}
  \Point{Linear Algebra​ and Numerical Analysis}
  \Point{Calculus of One Variable}
  \Point{Calculus of Many Variables}
  \Point{Discrete Mathematics for Computer Science​}
  \Point{Data Structures and Algorithms*}
  \Point{Introduction to Machine Learning*}
\end{Course}
\section{Technical Skills}
\Point{\textbf{Design and Analysis Software:} ​CREO 3.0; SolidWorks ; Abaqus (Proficient in all)}
\Point{\textbf{Programming Languages:} ​C/C++ (Proficient) ; Python (Proficient); Java (Familiar); MATLAB (Proficient) }
\Point{\textbf{Data Analysis:} ​R; Python; Microsoft Office Excel (Proficient in all)}
\Point{\textbf{Machine Learning:} Theano Library in Python (Familiar)}

\section{Positions of Responsibility}
\datedsubsection{Project Representative, National Service Scheme}{Jul 2015 - May 2016}
\Point{Was the Head of a 15 member team responsible for the NSS Project Education Via Blogging.}
\Point{Took up initiatives which led to a content generation of 90 articles over the year showing a 180\% increase in the number of articles as compared to the previous year.}
\Point{Was the best NSS Project in terms of the people who passed the NSS Course.}

\datedsubsection{Manager in Career Development Team ,International \& Alumni Relations}{May - Jul 2016}
\Point{Conducted Mock Interviews and GD's for final year students.}
\Point{Played important roles in conducting lectures by important alumni associated with big companies (in particular people like the VP of Flipkart, MD of Deutsche Bank).}
\Point{Ideated a new event called Mentor for Interns in 4 cities in which Alumnus would meet the students interning in Chennai, Bangalore, Delhi and Hyderabad and help them achieve the maximum out of their internships. Achieved huge success with around 200 students turning up for events}
\Point{Helped start a Consultancy Club for the Institute which would take up cases from various industries.}

\datedsubsection{Manager in Analytics Club}{May - Jul 2016}
\Point{Took up initiatives for the club which doubled the membership of the club from 100 to a little over 200 students.}
\Point{Conducted events for the club at least once every week to make sure that the students understood the basics of Data Analysis such as Regression, Classification, Text Mining, etc.}
\Point{Reached out to analytics start up companies to secure projects and internships for the club members.}

\section{Extra-Cirricular Activities}
\Point{Learning to play violin during the last 2 years.}
\end{document}